\documentclass[12pt,a4paper]{article}
\usepackage[italian]{babel}
\usepackage[utf8x]{inputenc}
 \title{Contratto}
\author{Mario Greco}
 \hyphenation{o-ri-gi-ne at-ti-vi-t piat-ta-for-ma} 
\begin{document}
  \maketitle

Si fornirà una piattaforma integrata per la promozione di servizi ed attività nell'ambito del turismo, dei beni culturali e delle denominazioni d'origine.
Lo scopo principale della piattaforma (cioè la promozione di eventi, prodotti, mostre, ecc...) verrà espletato fornendo la possibilità alle aziende operanti nei suddetti settori di registrarsi, mediante pagamento, alla piattaforma. La registrazione darà diritto alla pubblicazione di contenuti informativi pubblicamente accessibili.

Tale piattaforma sarà interamente amministrabile dal personale addetto della Provincia e fornirà strumenti per:
\begin{itemize}
 \item accettare, declinare, annullare le adesioni delle singole aziende, sulla base del loro eventuale stato di insolvenza e sulla base dell'attinenza delle attività dell'azienda con gli scopi della piattaforma
 \item definire le tariffe dei servizi offerti dalla piattaforma, creando eventualmente dei bundle (pacchetti) di servizi correlati a prezzo ridotto.
\end{itemize}

Le aziende aderenti potranno:
\begin{itemize}
 \item creare una semplice scheda relativa alle attività dell'azienda, utilizzando una struttura standard messa a disposizione dalla piattaforma stessa
 \item creare un sito web proprio di piccole dimensioni, sfruttando spazio di hosting acquistabile sulla piattaforma stessa
 \item vendere sul portale stesso i propri servizi e prodotti, attraverso strumenti di e-commerce (nel caso di privati), o entrando in contatto diretto con l'acquirente (nel caso di altre aziende) che potrà compilare un modulo-tipo di contratto da sottoporre all'azienda che offre i servizi
 \item fornire al pubblico strumenti per la registratione a flussi di syndication
\end{itemize}

Il pubblico potrà accedere alle informazioni della piattaforma attraverso varie modalità:
\begin{itemize}
 \item il sito Web principale della piattaforma
 \item Totem (colonnine, terminali, o simili) permanenti o temporanei, installati presso luoghi di interesse turistico-culturale, o presso eventi temporanei organizzati dalla Provincia
 \item la versione mobile del sito web della piattaforma
 \item un'applicazione per dispositivi mobili che fornirà un sottoinsieme dei servizi della piattaforma.
\end{itemize}

Inoltre il pubblico, attraverso i mezzi appena specificati, avrà la possibilità di:
\begin{itemize}
 \item visionare le informazioni pubblicate da ogni singola azienda
 \item lasciare un voto, un'opinione, un commento, o una recensione su una azienda o su uno dei prodotti/servizi venduti e consultare quelli lasciati dagli altri utenti
 \item condividere tali giudizi sui principali social networks
 \item acquistare on-line prodotti e servizi o sottoporre dei contratti all'attenzione delle aziende aderenti.

\end{itemize}

\end{document}
